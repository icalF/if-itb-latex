%--------------------------------------------------------------------%
%
% Hypenation untuk Bahasa Indonesia
%
% @author Petra Barus
%
%--------------------------------------------------------------------%
%
% Secara otomatis LaTeX dapat langsung memenggal kata dalam dokumen,
% tapi sering kali terdapat kesalahan dalam pemenggalan kata. Untuk
% memperbaiki kesalahan pemenggalan kata tertentu, cara pemenggalan
% kata tersebut dapat ditambahkan pada dokumen ini. Pemenggalan
% dilakukan dengan menambahkan karakter '-' pada suku kata yang
% perlu dipisahkan.
%
% Contoh pemenggalan kata 'analisa' dilakukan dengan 'a-na-li-sa'
%
%--------------------------------------------------------------------%

\hyphenation {
	% A
	%
	a-na-li-sa
	a-pli-ka-si
  %
  % B
	%
  ban-ding
	be-be-ra-pa
	ber-ge-rak
  %
  % C
	%
	ca-ri
  %
  % D
	%
	da-e-rah
	di-nya-ta-kan
	de-fi-ni-si
  %
  % E
	%
	e-ner-gi
	eks-klu-sif
  %
  % F
	%
	fa-si-li-tas
  %
  % G
	%
	ga-bung-an
  gam-bar
  %
  % H
	%
	ha-lang-an
  %
  % I
	%
	i-nduk
  %
  % J
	%
	ka-me-ra
	kua-li-tas
  %
  % K
	%
  ke-per-caya-an
  kus-tom-i-sasi
  %
  % L
	%
  %
  % M
	%
  meng-guna-kan
  %
  % N
	%
  %
  % O
	%
  %
  % P
	%
  pe-ning-kat-an
  %
  % Q
	%
  %
  % R
	%
  %
  % S
	%
  %
  % T
	%
  %
  % U
	%
  u-sul
  %
  % V
	%
  %
  % W
	%
  %
  % X
	%
  %
  % Y
	%
  %
  % Z
	%
}
