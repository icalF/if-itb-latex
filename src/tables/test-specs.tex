\begin {table}[h]
\begin{center}
\caption {Spesifikasi Lingkungan Pengujian}
    \begin{tabular}{|l|l|}

\hline
\rowcolor{gray!15}
Komponen & Spesifikasi \\
\hline

Sistem Operasi & Ubuntu 16.04.5 amd64 \\
\hline

CPU & Intel Core i7-7500U CPU @ 2.70GHz × 4 \\
\hline

\emph{Host} RAM & 16 GB \\
\hline

Media Penyimpanan & SSD SATA III 6 Gbps \\
\hline

GPU & NVIDIA GeForce 940MX \\
\hline

Arsitektur GPU & Maxwell \\
\hline

\emph{Compute Capability} & 5.0 \\
\hline

\emph{Dedicated Video} RAM & 2 GB \\
\hline

CUDA \emph{Runtime} & CUDA 8.0 r2 \\
\hline

C/C++ \emph{Compiler} & GCC 5.4.0 \\
\hline

\end{tabular}

\end{center}
\end{table}
