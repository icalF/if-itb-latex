\chapter{Simpulan dan Saran}

\section{Simpulan}
    Beberapa kesimpulan yang didapat dari pengerjaan tugas akhir ini adalah sebagai berikut.

    \begin{enumerate}

        \item 
        Peningkatan kinerja pencocokan string pada IDS Snort menggunakan GPU berhasil dilakukan dengan cara mengubah pendekatan algoritma yang digunakan agar memaksimalkan utilisasi GPU hingga 60-70\%.

        \item
        Implementasi dilakukan dengan berbasis pada rancangan \cite{lin2013}, yaitu dengan Aho-Corasick varian tanpa \emph{failure function} dan delegasi \emph{thread} per karakter.
        
        \item 
        Operasi pencocokan string dari kamus merupakan operasi yang \emph{memory bound} sehingga dilakukan beberapa teknik untuk mengurangi \emph{latency gap} antara akses I/O, memori dan komputasi GPU. Di antara teknik yang digunakan yaitu penggunaan \emph{pinned memory} pada \emph{buffer input} dan \emph{texture memory} pada kamus. Selain itu, digunakan juga penggunaan \emph{shared memory} untuk menampung \emph{buffer input} dan tabel transisi dari \emph{state} awal untuk menghemat \emph{fetch} pada memori global.

        \item
        Parametrisasi dapat dilakukan pada jumlah \emph{thread} tiap \emph{block} dan ukuran \emph{buffer input}. Kedua besaran ini saling berhubungan jumlah \emph{thread} 
        
    \end{enumerate}

\section{Saran}
    Pengembangan yang dapat dilakukan selanjutnya terkait tugas akhir ini adalah sebagai berikut:

    \begin{enumerate}

        \item
        Pencocokan dilakukan oleh beberapa MPSE \emph{instance} secara simultan dengan memanfaatkan \emph{semaphore}.

        \item
        Melakukan penghematan transfer 

        \item
        Jumlah alokasi yang dilakukan dapat dikurangi dan memori dapat di-\emph{reuse} untuk menghindari asd
        
    \end{enumerate}