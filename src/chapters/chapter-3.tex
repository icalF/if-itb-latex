\chapter{Analisis Permasalahan dan Solusi}

  % Pada bab ini akan dipaparkan analisis permasalahan yang ada dalam permasalahan pembangunan membangun sistem pencegahan intrusi berbasis GPU. Selain itu, akan dipaparkan juga solusi yang diajukan dalam bentuk langkah-langkah pekerjaan yang akan dilakukan.

  \section{Analisis Permasalahan}
  
    Sistem pencegah intrusi jaringan (NIPS) adalah sistem yang bertugas melakukan penyaringan paket yang dianggap berbahaya. NIPS bekerja dengan cara menangkap paket yang berjalan dalam jaringan untuk kemudian dianalisis. Hasil analisis kemudian ditindaklanjuti berdasarkan rule yang telah didefinisikan. Paket dapat ditolak atau dibatasi. Dalam beberapa kasus, bahkan aliran paket dapat diblok.

    Dampak dari pengecekan paket adalah adanya \emph{overhead} waktu respons. Lama \emph{overhead} tergantung dari waktu pencocokan pola. Pada NIPS, hal ini dapat membuat hasil analisis menjadi tidak akurat. Maka diperlukan metode untuk melakukan analisis dengan cepat.

    Hingga saat ini, kecepatan traffic sudah mencapai 30-50 Gbps.

    Berbagai penelitian telah dilakukan untuk mempercepat NIDS dan NIPS. Desain paling awal yaitu menggunakan desain konkuren dengan \emph{multithreading} pada CPU \parencite{multi2004}. Kemudian desain berbeda yang menggunakan GPU mulai diajukan \parencite{gnort2008}. Hasil yang didapat mampu mempercepat sistem hingga 5x dibandingkan CPU dengan harga yang sepadan \parencite{smith2009}.

    % Pencocokan dapat dilakukan secara stateful, ataupun tidak. Pencocokan yang berbasis stateful akan lebih susah untuk dicek secara paralel.

    Meski demikian, masih ada beberapa masalah terjadi dalam desain yang telah diajukan. Pencocokan paket masih menggunakan algoritma yang berbasis insi

    Desain yang dapat digunakan untuk memanfaatkan paralelitas dapat ditinjau dari berbagai sisi, seperti core \emph{multithreading}, asynchronous memory transfer, dan struktur penyimpanan.

  \section{Analisis Solusi}
  
    % Berdasarkan permasalahan-permasalahan yang muncul dalam rangka membangun sistem pencegahan intrusi berbasis GPU, terdapat rancangan solusi yang diajukan untuk menyelesaikan permasalahan tersebut dalam bentuk langkah-langkah penelitian yang akan dilakukan. Langkah-langkah tersebut di antaranya adalah sebagai berikut.

    Berdasarkan pengukuran yang dilakukan pada \parencite{kargus2012}, didapatkan bahwa sebagian besar beban pencocokan berada pada tahap  pencarian string pada payload paket. Pada tahapan ini, sebagian besar paket yang tidak terindikasi sebagai serangan akan diloloskan. Sehingga beban pencocokan pada tahap berikutnya, yaitu pencocokan option rule akan berkurang drastis. Maka, fokus dari solusi yang akan diajukan yaitu menambah metode-metode yang akan mempercepat kinerja pencocokan string paket pada NIPS.

    Eksperimen penggunaan GPGPU pada NIPS akan memodifikasi NIDPS Snort. Snort digunakan karena kode bersifat \emph{open-source} dan mudah untuk dikembangkan, sehingga pengembangan dan eksperimen tidak memerlukan biaya banyak dan dapt dieksplorasi secara mandiri. 

    Sedangkan pengembangan akan menggunakan platform GPGPU CUDA. Platform CUDA merupakan platform GPGPU yang dibuat pada GPU NVIDIA. Platform CUDA digunakan karena memiliki fitur \emph{pinned memory} yang lebih cepat diakses.

    \subsection{Implementasi Algoritma Pencocokan \emph{Signature} pada CUDA}

      Algoritma yang akan digunakan adalah \emph{Parallel Failureless Aho-Corasick} (PFAC). Perbedaan algoritma ini dibandingkan algoritma Aho-Corasick yang biasa adalah algoritma ini tidak menggunakan \emph{failure function}. digunakan untuk mengurangi jumlah sinkronisasi yang dilakukan tiap \emph{thread}. 

    \subsection{Transfer Payload Paket}

      Pengiriman dari host ke device dan sebaliknya memiliki \emph{latency} yang cukup besar. Dalam pengujian pada \parencite{gnort2008}, didapatkan bahwa \emph{latency} dari transfer payload memiliki runtime lebih dari 80\% runtime total. Sehingga untuk menyembunyikan \emph{latency}, pengiriman akan dilakukan per batch. Ukuran batch yang digunakan akan diuji dalam ukuran yang bervariasi. Mulai dari 32MB, 64MB, 128MB, dan 256MB.

    \subsection{Struktur Penyimpanan \emph{Signature}}

      Penyimpanan \emph{signature} mempengaruhi besar memori yang digunakan. Struktur yang digunakan akan menggunakan trie terkompresi. Struktur digunakan untuk memaksimalkan \emph{locality} sehingga mengurangi \emph{latency}. Selain itu, dengan menurunkan konsumsi memori, besar buffer yang dapat digunakan meningkat. Diharapkan, \emph{latency} dari pengiriman juga menurun.

    \subsection{Pengujian dan Pengambilan Hasil}

      Pengujian akan dilakukan dalam dua tahap, uji pencocokan dan \emph{stress-test}. Uji pencocokan akan dilakukan untuk menguji akurasi klasifikasi paket. Sedangkan \emph{stress-test} akan menguji ketahanan sistem dalam lingkungan yang \emph{real-time}.