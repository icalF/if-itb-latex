%!TeX root=../thesis.tex

\chapter{Implementasi dan Pengujian}

Pembahasan pada bab ini akan dibagi menjadi dua bagian. Bagian pertama dipaparkan tentang lingkungan dan detail implementasi. Sedangkan bagian kedua dibahas mengenai teknis dan hasil pengujian terhadap perangkat lunak yang sudah dibangun.

\section{Implementasi}

  Bagian implementasi dibagi menjadi 4 bagian. Bagian pertama dibahas mengenai lingkungan implementasi. Bagian kedua dibahas mengenai batasan implementasi. Bagian ketiga dibahas mengenai spesifikasi modul. Bagian terakhir berisi konfigurasi sistem deteksi intrusi.

  \subsection{Lingkungan Implementasi}

    Dalam proses implementasi digunakan sebuah laptop dan sejumlah perangkat lunak untuk menyelesaikan perangkat lunak yang dibangun. Detil spesifikasi perangkat keras dan lunak yang digunakan adalah seperti pada Tabel IV.1.
    
    \begin {table}[h]
\begin{center}
\caption {Spesifikasi Lingkungan Implementasi}
\begin{tabular}{|l|l|}

\hline
\rowcolor{gray!10}
Komponen & Spesifikasi \\
\hline

Sistem Operasi & Ubuntu 16.04.2 AMD 64-bit \\
\hline

CPU & Intel{\textregistered} Core{\texttrademark} i7-7500U CPU @ 2.70GHz × 4 \\
\hline

Host RAM & 16 GB \\
\hline

GPU & NVidia GeForce 940MX Compute Capablity 4.0 \\
\hline

Media Penyimpanan & \emph{Solid state drive} Samsung EVO 850 1TB \\
\hline

Dedicated Video RAM & 2 GB \\
\hline

CUDA Runtime & CUDA 8.0 r2 \\
\hline

G++ Compiler & GCC ??? \\
\hline

\end{tabular}
\end{center}
\end{table}


  \subsection{Batasan Implementasi}

    Implementasi dilakukan dengan batasan-batasan sebagai berikut.

    \begin{enumerate}

      \item
      Pencocokan menggunakan satu instance GPU \\
    
      \item
      Pencocokan hanya dilakukan satu kernel code \\
    
      \item
      Masukan paket hanya dilakukan dari berkas PCAP \\
    
%    \item
%    Konfigurasi dilakuakn

    \end{enumerate}

  \subsection{Spesifikasi Modul}
  
   \begin{enumerate}

      \item
      Komponen pembentukan kamus \\
    
      \item
      Komponen pencocokan \\
    
      \item
      Komponen penggabungan hasil \\

    \end{enumerate}

  \subsection{Konfigurasi}
    \blindtext

\section{Pengujian}

  \subsection{Tujuan Pengujian}
    Pengujian dilakukan untuk mendapatkan perbandingan kinerja antara modul yand dikembangkan dan modul yang ada dalam NIDS Snort. 

  \subsection{Skenario Pengujian}
  
    Pengujian dilakukan dengan membuat dua mesin, satu sebagai 
    Pengujian dilakukan dengan menerima masukan berupa berkas PCAP. Berkas didapatkan dari log traffic DEFCON ?? yang berlangsung pada ???. Dataset didapatkan dari ???.
    
  \subsection{Lingkungan Pengujian}
    
    Pengujian dilakukan pada mesin yang sama dengan lingkungan pengembangan.
    
    \begin {table}[h]
\begin{center}
\caption {Spesifikasi Lingkungan Implementasi}
\begin{tabular}{|l|l|}

\hline
\rowcolor{gray!10}
Komponen & Spesifikasi \\
\hline

Sistem Operasi & Ubuntu 16.04.2 AMD 64-bit \\
\hline

CPU & Intel{\textregistered} Core{\texttrademark} i7-7500U CPU @ 2.70GHz × 4 \\
\hline

Host RAM & 16 GB \\
\hline

GPU & NVidia GeForce 940MX Compute Capablity 4.0 \\
\hline

Media Penyimpanan & \emph{Solid state drive} Samsung EVO 850 1TB \\
\hline

Dedicated Video RAM & 2 GB \\
\hline

CUDA Runtime & CUDA 8.0 r2 \\
\hline

G++ Compiler & GCC ??? \\
\hline

\end{tabular}
\end{center}
\end{table}


  \subsection{Hasil Pengujian}
    \blindtext

  \subsection{Pembahasan}
    \blindtext