%!TeX root=../thesis.tex

\chapter{Implementasi dan Pengujian}

Pembahasan pada bab ini akan dibagi menjadi dua bagian. Bagian pertama dipaparkan tentang lingkungan dan detail implementasi. Sedangkan bagian kedua dibahas mengenai teknis dan hasil pengujian terhadap perangkat lunak yang sudah dibangun.

\section{Implementasi}

  Bagian implementasi dibagi menjadi 4 bagian. Bagian pertama dibahas mengenai lingkungan implementasi. Bagian kedua dibahas mengenai batasan implementasi. Bagian ketiga dibahas mengenai spesifikasi modul. Bagian terakhir berisi konfigurasi sistem deteksi intrusi.

  \subsection{Lingkungan Implementasi}

    Dalam proses implementasi digunakan sebuah komputer dan sejumlah perangkat lunak untuk menyelesaikan perangkat lunak yang dibangun. Detil spesifikasi perangkat keras dan lunak yang digunakan adalah seperti pada Tabel IV.1.
    
    \begin {table}[h]
\begin{center}
\caption {Spesifikasi Lingkungan Implementasi}
\begin{tabular}{|l|l|}

\hline
\rowcolor{gray!10}
Komponen & Spesifikasi \\
\hline

Sistem Operasi & Ubuntu 16.04.2 AMD 64-bit \\
\hline

CPU & Intel{\textregistered} Core{\texttrademark} i7-7500U CPU @ 2.70GHz × 4 \\
\hline

Host RAM & 16 GB \\
\hline

GPU & NVidia GeForce 940MX Compute Capablity 4.0 \\
\hline

Media Penyimpanan & \emph{Solid state drive} Samsung EVO 850 1TB \\
\hline

Dedicated Video RAM & 2 GB \\
\hline

CUDA Runtime & CUDA 8.0 r2 \\
\hline

G++ Compiler & GCC ??? \\
\hline

\end{tabular}
\end{center}
\end{table}


  \subsection{Batasan Implementasi}

    Implementasi dilakukan dengan batasan-batasan sebagai berikut.

    \begin{enumerate}

      \item Implementasi hanya mengerjakan komponen MPSE (\emph{multipattern search engine}) pada Snort

      \item Pencocokan menggunakan satu \emph{instance} GPU
    
      \item Pencocokan hanya dilakukan pada satu \emph{thread kernel code}
    
      \item Akuisisi \emph{payload} hanya dilakukan dari berkas PCAP

    \end{enumerate}

  \subsection{Spesifikasi Komponen}

    Implementasi dilakukan dengan membuat beberapa komponen, yaitu:

    \begin{enumerate}

      \item
      \emph{Handle} \\
      \emph{Handle} merupakan struktur data yang membungkus fungsionalitas pencocokan untuk dapat digunakan pada \emph{pipeline} Snort. \emph{Handle} berisi fungsi-fungsi yang terkait modul yang dikembangkan seperti menambahkan pola, pembentukan kamus, pencarian, dan mencetak informasi spesifik \emph{handle}. Fungsi-fungsi pada \emph{handle} akan memanggil modul lain yang terkait, misalnya fungsi pembentukan kamus akan memanggil komponen pembentukan kamus pada kode PFAC yang dibuat.

      \item
      \emph{State machine} \\
      \emph{State machine} merupakan kamus yang digunakan dalam pencocokan. \emph{State machine} dibentuk di memori \emph{host}. Pola yang telah dikumpulkan dalam \emph{list}, akan disusun ulang sehingga terurut secara alfabetis. Kemudian akan dilakukan penelusuran DFS (\emph{depth-first search}) untuk membangun indeks status.

      \item
      \emph{Kernel wrapper} \\
      \emph{Kernel wrapper} adalah \emph{wrapper} yang menyiapkan \emph{payload} sebelum \emph{kernel} dijalankan. \emph{Wrapper} akan menyalin \emph{payload} dari \emph{host buffer} ke \emph{device buffer}. Setelah persiapan selesai, \emph{wrapper} akan memanggil kode \emph{kernel} untuk menjalankan pencocokan.\clearpage
      
      \item
      \emph{Matcher kernel} \\
      \emph{Kernel} adalah kode GPU yang akan dijalankan untuk pencocokan. \emph{Kernel} hanya akan melakukan \emph{looping} terhadap nomor status berdasarkan tabel transisi yang telah dimuat. \emph{Looping} akan berhenti ketika status telah menuju status akhir maupun status \emph{dummy}.

      \item
      \emph{Reducer} \\
      \emph{Reducer} akan menggabungkan hasil pencocokan dari tiap \emph{thread} sehingga diperoleh jumlah total rule yang terdeteksi. \emph{Reducer} dipanggil setelah \emph{matcher} dijalankan. \emph{Reducing} akan dilakukan secara sekuensial pada CPU. Hasil \emph{reducing} akan dikembalikan ke fungsi \emph{search} milik \emph{handle} untuk dikembalikan lagi ke Snort.
      
    \end{enumerate}

  \subsection{Konfigurasi}

    Agar Snort dapat dijalankan dengan modul yang telah diintegrasikan, perlu dibuat pengubahan pada konfigurasi. Konfigurasi akan disuplai dari skrip lua. Diantara poin yang perlu diubah pada berkas konfigurasi yaitu:

    \begin{enumerate}
      \item Pencocokan dilakukan dengan modul PFAC yang telah selesai diintegrasikan.
      \item Modul \emph{daq (data acquisition)} yang digunakan hanya pcap-daq untuk dapat membaca berkas PCAP.
      \item \emph{Ruleset} diatur menunjuk ke lokasi \emph{rule} Talos.
      % \item \emph{Buffer} untuk \emph{payload} diatur sebesar 512.000.000 \emph{byte} atau sebesar 512 MB.
    \end{enumerate}

\section{Pengujian}

  \subsection{Tujuan Pengujian}
    Pengujian dilakukan untuk mendapatkan perbandingan kinerja antara modul yand dikembangkan dan modul yang ada dalam NIDS Snort. 

  \subsection{Skenario Pengujian}
  
    % Pengujian dilakukan dengan membuat dua mesin, satu sebagai 
    Pengujian dilakukan dengan menerima masukan berupa berkas PCAP. Berkas didapatkan dari \emph{log traffic} kompetisi DEFCON 25 yang berlangsung pada bulan Juli tahun 2017. Dataset didapatkan dari jaringan \emph{torrent} milik arsip DEFCON.
    
  \subsection{Lingkungan Pengujian}
    
    Pengujian dilakukan pada mesin yang sama dengan lingkungan pengembangan.
    
    \begin {table}[h]
\begin{center}
\caption {Spesifikasi Lingkungan Pengujian}
    \begin{tabular}{|l|l|}

\hline
\rowcolor{gray!15}
Komponen & Spesifikasi \\
\hline

Sistem Operasi & Ubuntu 16.04.5 amd64 \\
\hline

CPU & Intel Core i7-7500U CPU @ 2.70GHz × 4 \\
\hline

\emph{Host} RAM & 16 GB \\
\hline

Media Penyimpanan & SSD SATA III 6 Gbps \\
\hline

GPU & NVIDIA GeForce 940MX \\
\hline

Arsitektur GPU & Maxwell \\
\hline

\emph{Compute Capability} & 5.0 \\
\hline

\emph{Dedicated Video} RAM & 2 GB \\
\hline

CUDA \emph{Runtime} & CUDA 8.0 r2 \\
\hline

C/C++ \emph{Compiler} & GCC 5.4.0 \\
\hline

\end{tabular}

\end{center}
\end{table}


  \subsection{Hasil Pengujian}
    % \blindtext

  \subsection{Pembahasan}
    % \blindtext