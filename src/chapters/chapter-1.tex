\chapter{Pendahuluan}

Pada bab ini akan dibahas mengenai gambaran dasar dari pelaksanaan tugas akhir dalam bentuk penjelasan latar belakang yang mendasari pemilihan topik. Dari latar belakang tersebut, akan diurai kembali menjadi rumusan masalah, tujuan, batasan masalah, serta metodologi yang digunakan untuk keperluan Tugas Akhir ini.

\section{Latar Belakang}

SNORT merupakan sistem deteksi intrusi yang populer digunakan. SNORT menggunakan PCRE (Perl Compatible Regular Expression) untuk pencocokan pola paket. Pola paket akan diperbarui secara berkala dari komunitas penggiat keamanan. Dengan adanya sistem ini, pencocokan dapat dilakukan lebih cepat dan fleksibel. \\
Namun dengan makin banyaknya traffic yang lewat, IDS dengan performa yang lebih baik sangat dibutuhkan. Peningkatan performa yang biasa dilakukan dengan cara scale out. Penggunaan multi node maupun multicore processor telah dikembangkan untuk mempercepat performa IDS. \\
Metode-metode diatas dapat mempercepat pencocokan pola hingga 300\%. Meski demikian, performa masih terhambat latency antar core/node sehingga belum optimal. Dan juga kebanyakan arsitektur fokus melakukan scale out hanya pada satu aspek. Ini juga menyebabkan latency antar komponen juga turut memperlambat sistem. 


\section{Rumusan Masalah}

Rumusan Masalah berisi masalah utama yang dibahas dalam tugas akhir. Rumusan masalah yang baik memiliki struktur sebagai berikut:

\begin{enumerate}
    \item Penjelasan ringkas tentang kondisi/situasi yang ada sekarang terkait dengan topik utama yang dibahas Tugas Akhir.
    \item Pokok persoalan dari kondisi/situasi yang ada, dapat dilihat dari kelemahan atau kekurangannya. Bagian ini merupakan inti dari rumusan masalah.
    \item Elaborasi lebih lanjut yang menekankan pentingnya untuk menyelesaikan pokok persoalan tersebut.
    \item Usulan singkat terkait dengan solusi yang ditawarkan untuk menyelesaikan persoalan.
\end{enumerate}

Penting untuk diperhatikan bahwa persoalan yang dideskripsikan pada subbab ini akan dipertanggungjawabkan di bab Evaluasi apakah terselesaikan atau tidak.

\section{Tujuan}

Tugas akhir ini akan mencoba menjawab pertanyaan apa saja yang dapat dilakukan untuk memperkecil latency pada pemroses paralel pada IDS dan berapa besar kenaikan performa yang dapat dicapai.

\section{Batasan Masalah}

Fokus masalah yang dibahas pada tugas akhir ini yaitu pengembangan arsitektur yang efisien untuk pencocokan pola SNORT IDS menggunakan pemroses grafis (GPU).

\section{Metodologi}

Tahapan yang akan dilakukan selama pelaksanaan tugas akhir ini adalah sebagai berikut:


\section{Jadwal Pelaksanaan Tugas Akhir}

Tuliskan rencana kegiatan dan jadwal (dirinci sampai per minggu) mulai dari awal pelaksanaan Tugas Akhir I s.d. sidang tugas akhir berikut milestones dan deliverables yang harus diberikan. Jadwal ini dapat dibantu dengan membuat sebuah tabel timeline.
