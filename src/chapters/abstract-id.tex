\chapter*{ABSTRAK}
\addcontentsline{toc}{chapter}{ABSTRAK}

\begin{center}
\MakeTextUppercase{\textbf{\large{\thetitle}}}

Oleh

\MakeTextUppercase{\theauthor}
\end{center}
\medskip
\begin{spacing}{1.0}

Saat ini, banyak transaksi penting terjadi dunia maya. Keamanan informasi menjadi jaminan yang sangat penting agar data-data penting tidak bocor dan disalahgunakan. Sistem deteksi dan pencegahan intrusi telah dikembangkan. Namun, kecepatan deteksi intrusi belum secepat peningkatan kecepatan jaringan.

Solusi yang dapat digunakan yaitu penggunaan \emph{multithread} untuk pencocokan yang banyak secara paralel. Solusi \emph{multithreading} telah dikembangkan menggunakan CPU. Namun, CPU memiliki \emph{core} yang sangat terbatas. Alternatif lainnya adalah dengan menggunakan GPU. GPU memiliki kemampuan untuk membangkitkan banyak \emph{thread} sekaligus. GPU sangat cocok untuk melakukan komputasi sederhana dalam jumlah besar. 

Salah satu komponen penting dalam sistem deteksi intrusi yaitu pencocokan string. Beban pencocokan string ini seringkali menjadi \emph{bottleneck} dalam analisis paket jaringan. Tugas akhir ini akan mencoba melakukan eksperimen untuk melakukan implementasi pencocokan string pada sistem deteksi intrusi Snort menggunakan GPU.

Untuk dapat mempercepat proses analisis menggunakan GPU, translasi solusi yang ada tidaklah cukup. Operasi dalam GPU seringkali adalah operasi yang I/O \emph{bound} dan \emph{memory bound}. Diperlukan beberapa pengubahan seperti alokasi \emph{thread} yang berbeda, metode transfer memori \emph{host} dan \emph{device}, skema penampungan paket, dan pengubahan struktur data \emph{state machine}. Implementasi dapat menghasilkan \emph{speedup} yang signifikan hingga 3x lipat solusi \emph{multithreading} dengan CPU.

Kata kunci: \textit{pattern matching}, deteksi intrusi, GPU, optimasi, paralel.


\end{spacing}

\clearpage