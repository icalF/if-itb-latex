\chapter*{Kata Pengantar}
\addcontentsline{toc}{chapter}{KATA PENGANTAR}

Puji dan syukur penulis panjatkan ke hadirat Tuhan yang Maha Esa karena atas berkat dan pertolongan-Nya, penulis dapat menyelesaikan Tugas Akhir dan laporan tugas akhir yang berjudul "\thetitle" ini dengan baik. 

Dalam penyusunan Tugas Akhir ini penulis banyak mendapatkan masukan, kritik, dorongan, bantuan, bimbingan, serta dukungan baik secara fisik maupun moral dari berbagai pihak yang merupakan pengalaman yang berharga yang tidak dapat diukur secara materi dan dapat menjadi pembelajaran yang berharga dikemudian hari. Oleh karena itu dengan segala hormat dan kerendahan hati perkenankanlah penulis mengucapkan terima kasih kepada:

\begin{enumerate}
    \item Bapak Achmad Imam Kistijantoro, S.T., M.Sc., Ph.D. selaku pembimbing penulis atas ilmu yang diberikan selama bimbingan bimbingan maupun perkuliahan, kritik dan saran, serta dukungan moral yang diberikan selama proses pengerjaan tugas akhir.

    \item Bapak Yudistira Dwi Wardhana Asnar, Ph.D selaku penguji atas saran dan masukkannya sehingga membuat Tugas Akhir ini menjadi lebih baik.

    \item Kedua orang tua penulis. Terima kasih atas dukungan baik secara moral dan material sehingga penulis dapat melaksanakan Tugas Akhir ini hingga selesai dengan baik.
    
    \item Ibu Dr. Fazat Nur Azizah ST, M.Sc. selaku dosen Tugas Akhir yang telah memotivasi saya dan memberikan arahan dalam penyelesaian dan pengerjaan Tugas Akhir ini.
    
    \item Ibu Dr. Eng. Ayu Purwarianti, ST., MT. selaku dosen mentor Imagine Cup yang ikut membantu dan memberikan dukungan agar penulis dapat menyelesaikan pengerjaan Tugas Akhir.
    
	\item Bapak Dr.techn. Saiful Akbar ST, MT. dan Bapak Achmad Imam Kistijantoro, ST, M.Sc, Ph.D selaku Ketua Program Studi dari Teknik Informatika dan Sistem dan Teknologi Informasi yang ikut mendukung penulis dalam menyelesaikan Tugas Akhir ini.
    
    \item Seluruh dosen, karyawan dan civitas program studi Teknik Informatika, Institut Teknologi Bandung.
    
    \item Rekan-rekan yang telah penulis minta bantuan dalam penyelesaian administrasi Tugas Akhir ini disaat penulis berhalangan terutama Muhamad Visat Sutarno dan Fiqie Ulya Sidiastahta
    
    \item Rekan-rekan laboratorium basis data yang saling mendukung dalam menyelesaikan Tugas Akhir ini pada khususnya Vanya Deasy Safrina, Albert Tri Adrian, Marco Orlando, Fiqie Ulya Sidiastahta, dan Wilhelmus Andrian.
    
    \item Rekan-rekan seperjuangan Teknik Informatika yang menamakan dirinya Happy Anti Wacana yang selalu mendukung penulis untuk mengerjakan Tugas Akhir.
    
    \item Rekan-rekan dari Binary 2013 dan HMIF yang telah memberikan dukungan dan bantuan dalam pengerjaan Tugas Akhir.
    
    \item Pihak-pihak lain yang tidak dapat disebutkan satu-persatu.
\end{enumerate}

Penulis menyadari bahwa Tugas Akhir ini masih jauh dari sempurna serta memiliki banyak kekurangan. Oleh karena itu, penulis sangat terbuka dalam menerima kritik dan saran yang membangun untuk Tugas Akhir ini. Semoga Tugas Akhir ini dapat bermanfaat bagi pembaca.

\begin{flushright}
Bandung, September 2018 \\
\vspace{25mm}
Penulis
\end{flushright}

